\documentclass[10pt,a4paper]{article}
\usepackage[utf8]{inputenc}
\usepackage[ngerman]{babel}
\usepackage[T1]{fontenc}
\usepackage{amsmath}
\usepackage{amsfonts}
\usepackage{amssymb}
\usepackage{graphicx}
\usepackage{lmodern}
\usepackage{physics}
\usepackage[left=1cm,right=1cm,top=2cm,bottom=1.5cm]{geometry}
\usepackage{siunitx}
\usepackage{fancyhdr}
\usepackage{enumerate}
\usepackage{mhchem}
\usepackage{mathtools}
\usepackage{graphicx}
\usepackage{float}
\usepackage{xcolor}
\usepackage{mdframed}
\usepackage{csquotes}
\usepackage{trfsigns}
\usepackage{capt-of}
\usepackage{ulem}
\usepackage{tikz}
\usepackage[most]{tcolorbox}
\usepackage{empheq}
\usepackage{parskip}
\usepackage{enumitem}
\usepackage{esdiff}
\usepackage{commath}
\usepackage[dvipsnames]{xcolor}
\usepackage{minted}


\sisetup{locale=DE}
\sisetup{per-mode = symbol-or-fraction}
\sisetup{separate-uncertainty=true}
\DeclareSIUnit\year{a}
\DeclareSIUnit\clight{c}
\mdfdefinestyle{exercise}{
	backgroundcolor=black!10,roundcorner=8pt,hidealllines=true,nobreak
}

\newtcbox{\mymath}[1][]{%
  nobeforeafter, math upper, tcbox raise base, enhanced, colback=yellow!5!white, colframe=red!50!black, sharp corners, boxrule=1pt, boxsep=0pt, left=0pt, right=0pt, top=2pt, bottom=2pt, #1}
\newtcolorbox{Bedingungsbox}[1][]{colback=gray!20, colframe=gray!50, sharp corners, boxrule=0.7pt, #1}

\setlength{\parindent}{0pt}
\definecolor{sub}{HTML}{cde4ff}
\definecolor{main}{HTML}{5989cf}
\tcbset{
    sharp corners,
    colback = white,
    before skip = 0.2cm,  
    after skip = 0.5cm    
}

\newtcolorbox{Definitionsbox}{
    colback = sub, 
    colframe = main, 
    boxrule = 0pt, 
    leftrule = 6pt 
}

\begin{document}
\twocolumn
\pagestyle{fancy}
\lhead{Angewandte Mathematik: Differentialgleichungen \\ Zusammenfassung KME V6d}
\rhead{\today \\ Jannick Riond}

\section{Definition}
\begin{Definitionsbox}
  Unter einer gewöhnlichen Differentialgleichung $n$-ter Ordnung verstehen wir eine Gleichung, in der Ableitungen einer unbekannten Funktion $y = y(x)$ bis zur $n$-ten Ordnung auftreten.
\end{Definitionsbox}
\begin{mdframed}
  \textbf{Beispiel:} Differentialgleichung 1. Ordnung
  \begin{equation*}
    y' = sin(x)
  \end{equation*}
  \\[10pt]
$y' = y^{(1)}$: Lagrange-Schreibweise\\[5pt]
$\diff{y}{x}$: Leibnitz-Schreibweise\\[5pt]
$\dot{y}$: Newton-Schreibweise
\end{mdframed}

\subsection{Autonome Differentialgleichung}
\begin{Definitionsbox}
  Eine \textbf{Autonome Differentialgleichung} ist eine spezielle Art von Differentialgleichung, bei der die Ableitung nicht von $x$, also der unabhängigen Variable abhängt, sondern nur vom aktuellen Zustand der Funktion selbst.
\end{Definitionsbox}
\begin{mdframed}
  \textbf{Beispiel:} Autonome Differentialgleichung\\
  Die Gleichung
  \begin{equation*}
    \diff{y}{x} = y(x)
  \end{equation*}
  ist autonom, weil auf der rechten Seite $x$ nicht explizit vorkommt und somit nicht von $x$ abhängig ist.
\end{mdframed}

\section{Lösung einer Differentialgleichung}
\subsection{Definition der Lösung}
\begin{Definitionsbox}
  Eine Funktion $y = y(x)$ heisst \textbf{Lösung der Differentialgleichung}, wenn sie mit ihren Ableitungen die Differentialgleichung identisch erfüllt.
\end{Definitionsbox}
Unter der allgemeinen Lösung einer Differentialgleichung versteht man grundsätzlich eine Lösungsmenge, die mithilfe der Integrationskonstante $C \in \mathbb{R}$ beschrieben wird.
\begin{mdframed}
  \textbf{Beispiel:} Allgemeine Lösung
  \begin{equation*}
    y = - cos(x) + C
  \end{equation*}
  $C$: Integrationskonstante
\end{mdframed}
Eine spezielle Lösung erhalten wir durch Einsetzen eines bestimmten Werts für diesen Paramter $C$.
\begin{mdframed}
  \textbf{Beispiel:} Spezielle Lösung
  \begin{equation*}
    y = - cos(x) + 3
  \end{equation*}
\end{mdframed}

\subsection{Graphische Lösung: Das Richtungsfeld}
\begin{Definitionsbox}
  Ein \textbf{Anfangswertproblem} besteht aus einer Differentialgleichung und einer Anfangsbedingung.
\end{Definitionsbox}

\begin{equation*}
  \begin{cases}
    \begin{aligned}
      y' &= f(x,y) \quad &\text{Differentialgleichung}\\ 
      y(x_0) &= y_0 \quad &\text{Anfangsbedingung}
    \end{aligned}
  \end{cases}
\end{equation*}

\begin{Definitionsbox}
  Das \textbf{Richtungsfeld} einer Differentialgleichung ist die Gesamtheit der Richtungselemente, welche die Steigung der Lösungskurve, die durch diesen Punkt verlaufen, repräsentieren.
\end{Definitionsbox}

\begin{mdframed}
  \textbf{Beispiel:}
  \begin{center}
  \begin{tikzpicture}
      \begin{scope}[scale=0.8]
          \def\dx{0.5}; % x-spacing for ticks
          \def\dy{0.5}; % y-spacing for ticks
          \def\sx{-5};  % lower bound for x values
          \def\sy{-5};  % lower bound for y values
          \def\ex{5};   % upper bound for x values
          \def\ey{5};   % upper bound for y values
          \def\l{0.2};  % length of HALF of a segment
          
          % draw grid
          \draw[Gray, thin] (-5,-5) grid (5,5);
          \draw[thick,latex-latex] (-5.5,0) -- (5.5,0);
          \draw[thick,latex-latex] (0,-5.5) -- (0,5.5);
          
          \foreach \i in {-5,...,-1,1,2,...,5} {
            \draw[thick] (\i,0)--(\i,-0.1);
            \node[below] at (\i,-0.1) {\i};
            \draw[thick] (0,\i)--(-0.1,\i);
            \node[left] at (-0.1,\i) {\i};
          }
          
          % draw slope ticks:
          \foreach \x  in {-5,-4.5,...,5} {
            \foreach \y  in {-5,-4.5,...,-0.5,0.5,1,...,5} {
              \pgfmathsetmacro{\m}{\x-\x/\y};
              \pgfmathsetmacro{\k}{\l/sqrt(1+\m*\m)};
              \pgfmathsetmacro{\h}{\k*\m};
              \draw[ thick, Emerald] (\x,\y) -- (\x+\k,\y+\h);
              \draw[ thick, Emerald] (\x,\y) -- (\x-\k,\y-\h);
            }
          }
          
          % Euler's method
          \pgfmathsetmacro{\lasty}{2};
          \foreach \x in {2.2,2.4,...,3.8} {
            \pgfmathsetmacro{\thisx}{\x};
            \pgfmathsetmacro{\lastx}{\x-0.2};
            \pgfmathsetmacro{\f}{\lastx-\lastx/\lasty};
            \pgfmathsetmacro{\thisy}{\lasty + 0.2*\f};
            \draw[ultra thick, Orange] (\lastx,\lasty) -- (\thisx,\thisy);
            \global\let\lasty = \thisy;
          }
          \pgfmathsetmacro{\lasty}{2};
          \foreach \x in {1.8,1.6,...,-4} {
            \pgfmathsetmacro{\thisx}{\x};
            \pgfmathsetmacro{\lastx}{\x+0.2};
            \pgfmathsetmacro{\f}{\lastx-\lastx/\lasty};
            \pgfmathsetmacro{\thisy}{\lasty - 0.2*\f};
            \draw[ultra thick, Orange] (\lastx,\lasty) -- (\thisx,\thisy);
            \global\let\lasty = \thisy;
          }
          \draw[Orange,fill=white,thick] (2,2) circle (0.1);
          \end{scope}
          
  \end{tikzpicture}
\end{center}
  \begin{equation*}
    \begin{cases}
     \begin{aligned}
        y' &= x - \frac{x}{y}\\ 
        y(2) &= 2
      \end{aligned}
  \end{cases}
\end{equation*}
\\
\indent Isoklinen: $y = 1$ und $x = 0$
\end{mdframed}

\begin{Definitionsbox}
  \textbf{Isoklinen} sind Linien im Richtungsfeld, die gleiche Werte, also die gleiche Steigung, haben.
\end{Definitionsbox}

\newpage
\section{Numerik}
\subsection{Euler-Verfahren}
Das Euler-Verfahren ist ein numerisches (also näherungsweises) Verfahren zur Approximation der unbekannten Lösungskurve eines allgemeinen Anfangswertproblems:
\begin{equation*}
  \begin{cases}
    \begin{aligned}
      y' &= f(x,y)\\ 
      y(x_0) &= y_0
    \end{aligned}
  \end{cases}
\end{equation*}
Vom Ausgangspunkt P($x_0$,$y_0$) wird schrittweise mit der gewählten Schrittweite $h$ nach vorne gearbeitet, wobei $n \in \mathbb{N}$ und $h >0$.
\begin{align*}
  x_{n+1} &= x_n + h\\
  y_{n+1} &= y_n + hf(x_k,y_k)
\end{align*}

Allerdings entsteht dabei ein Fehler $e_k$, der mit zunehmendem Abstand zum Ausgangspunkt grösser wird. Der Fehler an der Stelle $x_k$ ist einfach die Differenz der beiden Werte:
\begin{equation*}
  e_k := y(x_k) - y_k
\end{equation*}

\subsection{Grafische Veranschaulichung}
Das Euler-Verfahren lässt uns die Funktionskurve erahnen. Je kleiner $h$, desto genauer wird die Approximation, jedoch steigt dadurch auch der Rechenaufwand, da wir noch mehr Punkte ausrechen müssen.
\begin{mdframed}
  \textbf{Beispiel:} $y' = x+y$, $x_0 = 0$, $y_0 = 1$, $h = 0.25$
  \begin{center}
    \begin{tikzpicture}
      \begin{scope}[scale=0.8]
        \def\dx{0.5}
        \def\dy{0.5}
        \def\sx{-5}
        \def\sy{-5}
        \def\ex{5}
        \def\ey{5}
        \def\l{0.2}
    
        % Draw grid and axes
        \draw[Gray, thin] (-5,-5) grid (5,5);
        \draw[thick,latex-latex] (-5.5,0) -- (5.5,0);
        \draw[thick,latex-latex] (0,-5.5) -- (0,5.5);
    
        % Ticks and labels
        \foreach \i in {-5,...,-1,1,2,...,5} {
          \draw[thick] (\i,0)--(\i,-0.1);
          \node[below] at (\i,-0.1) {\i};
          \draw[thick] (0,\i)--(-0.1,\i);
          \node[left] at (-0.1,\i) {\i};
        }
    
        % Vector field for y' = x + y
        \foreach \x in {-5,-4.5,...,5} {
          \foreach \y in {-5,-4.5,...,-0.5,0.5,1,...,5} {
            \pgfmathsetmacro{\m}{\x+\y}
            \pgfmathsetmacro{\k}{\l/sqrt(1+\m*\m)}
            \pgfmathsetmacro{\h}{\k*\m}
            \draw[thick,Emerald] (\x,\y) -- (\x+\k,\y+\h);
            \draw[thick,Emerald] (\x,\y) -- (\x-\k,\y-\h);
          }
        }
    
        % Euler's method: h = 0.25, starting at (0,1), n = 10 points
        \pgfmathsetmacro{\x}{0}
        \pgfmathsetmacro{\y}{1}
        \draw[Orange,fill=white,thick] (\x,\y) circle (0.1); % initial point
    
        \foreach \i in {1,...,6} {
          \pgfmathsetmacro{\f}{\x+\y}
          \pgfmathsetmacro{\xnew}{\x + 0.25}
          \pgfmathsetmacro{\ynew}{\y + 0.25*\f}
          \draw[ultra thick, Orange] (\x,\y) -- (\xnew,\ynew);
          \draw[Orange,fill=white,thick] (\xnew,\ynew) circle (0.08);
          \xdef\x{\xnew}
          \xdef\y{\ynew}
        }
    
        % Exact solution: y = 2e^x - x - 1
        \draw[thick, Purple, domain=-0.5:1.3, samples=100, smooth] 
          plot (\x, {2*exp(\x) - \x - 1});
      \end{scope}
    \end{tikzpicture}
    \end{center}

    \begin{itemize}
      \item Das Euler-Verfahren in \textcolor{Orange}{Orange}.
      \item Die exakte Lösung in \textcolor{Violet}{Violet}.
      \item Das Richtungsfeld in \textcolor{Emerald}{Grün (Emerald)}.
    \end{itemize}
\end{mdframed}

\newpage
\subsubsection{Exkurs Programmierung des Euler-Verfahrens}
\paragraph{Python}
\begin{minted}{python3}
  def euler (f, x0, y0, h, n)
    x = x0
    y = y0
    result = [(x, y)]
    for i in range(n):
      x += h
      y += h * f(x, y)
      result.append((x, y))
    return result
\end{minted}

\section{Separation}
Bei diesem Verfahren wird die Differentialgleichung so umgeformt, dass man sie anschliessend durch integrieren lösen kann.
\begin{Definitionsbox}
  Eine gewöhnliche Differentialgleichung 1. Ordnung heisst \textbf{seperabel}, wenn sie sich auf folgende Form bringen lässt:
  \begin{equation*}
    y' = h(y) \cdot g(x) = h(y(x)) \cdot g(x)
  \end{equation*}
\end{Definitionsbox}

\subsection{Lösungsvorgehen}
\begin{enumerate}
  \item $y' = \diff{y}{x}$ einfügen
  \item Seperation von $y$ und $x$
  \item Integrieren (links nach $y$ und rechts nach $y$)
  \item Auflösen nach $y$
\end{enumerate}
Wobei der 4. Schritt nicht immer möglich ist $\Rightarrow$ Numerisches Verfahren!

\begin{mdframed}
  \textbf{Beispiel:} Separation\\
  Differentialgleichung: $y' = -2 \cdot cos(x) \cdot y(x)$, wobei $y \neq 0$
  \begin{align*}
    \diff{y}{x} &= -2 \cdot cos(x) \cdot y(x) \qquad |\cdot \dif x \quad |\cdot \frac{1}{y}\\
    \int \big(\frac{1}{\textcolor{blue}{y}}\big) \dif \textcolor{blue}{y} &= \int \big(-2 \cdot cos(\textcolor{red}{x})\big) \dif \textcolor{red}{x}\\
    \ln(|y|) &= -2 \cdot \sin(x) + c_1 \qquad \quad | \, e^{()}\\
     |y| &= e^{-2 \cdot \sin(x) + c_1} \qquad \quad | \, c_2 := e^{c_1}\\
     |y| &= e^{-2 \cdot \sin(x)} \cdot c_2\\
     y(x) &= e^{-2 \cdot \sin(x)} \cdot c \, \Rightarrow \text{Allgemeine Lösung}
  \end{align*}
\end{mdframed}

\subsection{Singuläre Lösung}
Eine singuläre Lösung ist eine Lösung der Differentialgleichung, die beim normalen Herleiten nicht auftaucht.
\begin{mdframed}
  \textbf{Beispiel:} Singuläre Lösung\\
  Differentialgleichung: $y' = -2 \cdot cos(x) \cdot y(x)$
  \begin{align*}
     y' &= -2 \cdot cos(x) \cdot y(x)\\
     y(x) &= 0 \, \Rightarrow \text{Singuläre Lösung}
  \end{align*}
\end{mdframed}

\end{document}